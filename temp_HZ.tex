
% a paper template based on Elsivier

\documentclass[preprint,12pt]{elsarticle}
\usepackage{bm}
\usepackage{amssymb}
% \usepackage[cmintegrals]{newtxmath}
% \usepackage{scicite}
\usepackage[colorlinks,linkcolor=red]{hyperref}
\usepackage{natbib}
\journal{Journal of Fluid Mechanics}

\begin{document}

\begin{frontmatter}


% this is the title of the article
\title{\textbf{Title of the Research}}


% this is the authors list
\author{Hanfeng Zhai$^{1,2,\ast}$}

\address{$^{1}$Department of Mechanics, Shanghai University, \\SHANGDA Rd., Shanghai 200444, China}
\address{$^{2}$Second Affiliation, University of Whatever, \\Address., State 000000, Country} 

% this is the information of the corresponding author
\address{$^{\ast}$\rm To whom correspondence should be addressed; E-mail:  frankzhai0@gmail.com}

% put your abstract here
\begin{abstract}

Here is the abstract of the research. In this part, you can put your method and representative results in simple sentences.

\end{abstract}

\begin{keyword}Keyword A; Keyword B; Keyword C

\end{keyword}

\end{frontmatter}
%---------------------------------------------%



% this is the introduction part
\section{Introduction}

In the introduction part, you need to provide a overview and inspiration of your research.

%---------------------------------------------%


\section{Method}
\subsection{Method I}



Here we present a method for solving the presented problem; for example, we can write an equation here:
$2x+3y=34$. \\
\indent we can also write an equation in such a form:
\begin{equation}
    \mathbf{F} = m\mathbf{a}\label{01}
    \end{equation}
\indent We can cite the euqation as (\ref*{01}).

You can also include an algorithm in the form of
\begin{center}
    \includegraphics[scale = 1.1]{algorithm.pdf} \\ 
\end{center}
\subsection{Method II}

Another approach for the research work. You can include a figure in your research in the form of:

\begin{center}
    \includegraphics[scale = 0.5]{figure.pdf} \\ 
\end{center}
\begin{center}
\noindent {\bf Fig. 1.} Caption for figures.
\end{center}

%---------------------------------------------%



% this is the results of your research
\section{Result and discussions}
Here is your final results including many beatiful figures and plots. You can cite the reference like this [\citealp{01}].


%---------------------------------------------%




% this is the conclusion of your research
\section{Conclusion}
Here you briefly summarize your previous works and present some basic conclusions.


%---------------------------------------------%

% this is the conclusion of your research
\section*{Acknowledgement}
The author A would like to thank person A, person B, and person C for the valuable discussions. The authors declare no conflicts of interests.
% \bibliography{1.bib}
%---------------------------------------------%

\bibliographystyle{IEEEtran}
\bibliography{1.bib}


% this is the references part
\begin{thebibliography}{99}



 \bibitem{01} X. Chen, D. Duan, and G.E. Karniadakis. Learning and meta-learning of stochastic advection–diffusion–reaction systems from sparse measurements. \emph{European Journal of Applied Mathematics}. \textbf{2020}, 1-24.
 \bibitem{02} A, B, and C. Title of the work. \emph{Journal of Whatever}. \textbf{year}, number
 \bibitem{03}  
 \bibitem{04}
 \bibitem{05}
 \bibitem{06}
\end{thebibliography}

%---------------------------------------------%




\section*{Appendix. Code and supplementary data}
The code for running \({Algorithm\,1}\), \({Algorithm\,2}\), and \({Program\,1}\) is uploaded on \url{www.hanfengzhai.net}.

\section*{Appendix. Proof for Theorem I}
\indent You can add your proof for the theory given in the methods to make your reasoning more clear. 
In this section, you can also give additonal equations, such as:
\begin{equation}
    \mathbf{\sigma} = \rm E \mathbf{\epsilon}\label{02}
    \end{equation}


\end{document}


